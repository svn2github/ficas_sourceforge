%% Oh Emacs, this is a -*- Makefile -*-, so give me tabs.
\documentclass{article}
\usepackage{axiom}

\title{\$SPAD/src/algebra Makefile}
\author{Timothy Daly}

\begin{document}
\maketitle

\begin{abstract}
\end{abstract}
\eject

\tableofcontents
\eject


\section{The bootstrap problem\dots}

The current Axiom compiler compiles each constructor (category, domain
or package) separately.  In order to perform type checking Axiom
needs type information from other constructors.  To collect
type information about a constructor Axiom must compile it.
This creates a bootstrap problem: in order to compile a constructor
one must first compile many other constructors.

In principle the bootstrap problem could be avoided by organizing algebra
into layers, so that layer $0$ is independent of other layers and
each higher layer depends only on lower layers.  However, the current
Axiom algebra contains a lot of cyclic dependencies: for example, a substantial
part of algebra depends on [[Integer]], but [[Integer]] has a rich structure
and depends on many categories which in turn depend on [[Integer]].

\section{\dots and its solution}


In the past Axiom avoided significant part of bootstrap problem
simply by using type information collected in the past
and stored in databases [[interp.daase]] and [[category.daase]].
This solution meant that any attempt to change the signature of an exported
operation was likely to cause build failure.

Newer bootstrap procedure uses different approach.  It proceeds
in several phases.  After each phase of bootstrap we dump databases.
The databases dumped in a given phase are used in the next phase.
To limit number of phase (and also number of dumped databases)
in each phase we compile several files in a single image.

Using the type information stored in databases solves only part of
the bootstrap problem: internally, Axiom types are represented by
executable code and sometimes Axiom needs to load actual compiled
code.  The exact rules telling when Axiom needs compiled code
and when it is enough to have info in databases are unknown.
Axiom needs to load categories given as arguments to [[Join]],
because [[Join]] is implemented as a runtime operation.  It
seems that Axiom also needs to load a category given as an argument
to a constructor if this category itself has arguments.

During normal compilation Axiom tries to optimize (i.e., expand inline)
calls to operations from core domains, which are listed in the
[[$optimizableConstructorNames]] variable.  To do this, Axiom
has to load these domains.  In the past, this optimization required
keeping compiled Lisp corresponding to a collection of categories and
domains sufficient to cut cycles involving core domains.
Currently Axiom has a special flag [[$bootstrapDomains]] to
disable this optimization.

Even after using the [[$bootstrapDomains]] flag there are still
cyclic load dependencies caused by constructor arguments, but
those problem are resolved using the [[$bootStrapMode]] flag,
which causes relaxed type checking and skips executable parts
of categories.

The final process looks as follows: first, categories are
topologically sorted with respect to [[Join]] --  recall that arguments to
[[Join]] must be compiled before a category using [[Join]].
This order is slightly tweaked to avoid problems due to
constructor parameters.

In the first
phase we compile several fake categories and domains.  The only
purpose of using them is to inform Axiom that they are valid types
and what arity they have -- in effect they serve as forward
declarations.  We also collect all abbreviations in a single file
and read it into Axiom before compilation.  Abbreviations inform
Axiom that the corresponding names are constructors, which is
important to get correct parse.  

Once fake definitions are compiled we can compile categories
using [[$bootStrapMode]] set to [[t]].  This finishes phase 1. 

In phase 2 we recompile several categories and core domains
(still using [[$bootStrapMode]] set to [[t]]).  This phase
removes most of the traces of fake definitions made in phase 1.
We also compile several domains.

In phases 3, 4, and 5 we compile rest of domains and packages.

In principle it should be possible to use smaller number of
phases, but compile time grows nonlinearly with number of
files compiled in a single image, so we save time splitting
compilation into more phases.  Also, recompiling few core
domains in phase 2 does not work, so we recompile them at the
beginning of phase 3.

In the first five phases use [[$bootStrapMode]] set to [[t]].
For domains and packages it results in object code which
can not be used in normal compilation.  However, the main
result of phase 5 is database containing almost complete
type information.  Almost, because operation lists
are missing (this information is collected only
during normal compilation).
Phase 5 also makes compiled code of categories available to
phase 6.

In phase 6 categories and core domains are
compiled setting [[$bootstrapDomains]] to [[t]].  We have
to compile categories because categories may have associated
default domains which also may be loaded.  We also compile a
few domains which are needed for unknown reasons -- if we
skip them, the build will fail.  After phase 6 we obtaion
databases which still miss some information (most operation lists)
but allow compiling the rest of algebra.

In the phase 7 (stage 3 of previous bootstrap) we
compile categories and core domains in normal mode.
Next, we compile the bulk of algebra in alphabetical order.
More precisely, we let make choose the order, but the list we give is
sorted alphabetically, and make processes it in order.
We skip a few files becase thier compilation would fail.
After this step we obtaion databases which contain compilete
information about almost all of the algebra (it misses
information about a skipped files).

Finally we need two extra phases. In phase 8 we compile
a few files skipped in phase 7 and obtain a complete databse.
In the final phase we recompile several files which seem to be
miscompiled in previous steps (presumably due to missing database
entires).

\section{Adding new constructors}

Adding a new constructor is different from the above process.  It is
easier because a new constructor can freely use functions provided
by standard algebra.  On the other hand, information about new constructors
is absent from Axiom databases, so cyclic dependencies between new
constructors have no easy solution.  However if one is adding
mutually dependent constructors without cyclic dependencies
the solution is easy:  compile all dependent constructors in
a single Axiom image, using appropriate order.  The [[guess-pkg]]
target illustrates this method.

NOTE: If you add new constructors to this file you must also update

[[src/algebra/exposed.lsp.pamphlet]]

otherwise the new constructors won't be loaded by the interpreter when needed.

\section{Bootstrap procedure and variables}

The list [[SPADLIST]] below was produced using the following Perl
progam on the list of unpacked [[.spad]] files (as given by [[ls *.spad]]).
\begin{verbatim}
$j=0;
$cl = "     ";
while(<>) {
     chop;
     $l = $_;
     $l =~ s/\.spad/ /;
     $cl .= $l;
     if ($j == 5) {
         print $cl, "\\\n";
         $cl = "     ";
         $j=0;
     } else {
         $j++;
     }
}
$cl =~ s/ $//;
print $cl, "\n";
\end{verbatim}

<<bootstrap variables>>=

SPADLIST=\
     A1AGG ABELGRP ABELMON ABELSG ACF ACFS \
     ACPLOT AF AGG AHYP ALAGG ALGEBRA \
     ALGFACT ALGFF ALGMANIP ALGMFACT ALGPKG ALGSC \
     ALIST AMR ANON AN ANTISYM ANY1 \
     ANY APPLYORE APPRULE ARR2CAT ARRAY12 ARRAY1 \
     ARRAY2 \
     ASSOCEQ ASTACK ATRIG ATTREG AUTOMOR \
     BALFACT BASTYPE BBTREE BEZOUT BGAGG \
     BINARY BINFILE BITS BMODULE BOOLEAN BOP1 \
     BOP BOUNDZRO BRAGG BRILL \
     BSTREE BTAGG BTCAT BTOURN BTREE CABMON \
     CACHSET CARD CARTEN2 CARTEN CCLASS CDEN \
     CFCAT CHARNZ CHARPOL CHAR CHARZ CHVAR \
     CINTSLPE CLAGG CLIF CLIP CMPLXRT COLOR \
     COMBF COMBINAT COMBOPC COMM COMMUPC \
     COMPCAT COMPFACT COMPLEX2 COMPLEX COMPLPAT COMPPROP \
     COMRING CONTFRAC COORDSYS CPIMA CPMATCH CRAPACK \
     CRFP CSTTOOLS CTRIGMNP CVMP CYCLES CYCLOTOM \
     DBASE DBLRESP DDFACT \
     DECIMAL DEFINTEF DEFINTRF DEGRED DEQUEUE DERHAM \
     DFINTTLS DFLOAT DFSFUN DHMATRIX DIAGG DIFEXT \
     DIFRING DIOPS DIOSP DIRPCAT DIRPROD2 DIRPROD \
     DISPLAY DIVRING DLAGG DLIST DLP DMP \
     DPMM DPMO DPOLCAT DQAGG DRAWCFUN DRAWCURV \
     DRAWCX DRAWHACK DRAWPT DRAW DROPT0 DROPT1 \
     DROPT DSMP DVARCAT EAB \
     EF EFSTRUC EFULS EFUPXS ELAGG ELEMFUN \
     ELFUTS ELTAB ELTAGG EMR ENTIRER EP \
     EQ2 EQ EQTBL ERROR ES1 ES2 \
     ES \
     EUCDOM EVALAB EVALCYC EXIT EXPEXPAN EXPR2 \
     EXPR2UPS EXPRODE EXPR EXPRTUBE EXPUPXS FACTFUNC \
     FACUTIL FAGROUP FAMONC FAMONOID FAMR FARRAY \
     FAXF FCOMP FCPAK1 FC FDIV2 FDIVCAT \
     FDIV FEVALAB FEXPR FFCAT2 FFCAT FFCGP \
     FFCG FFCGX FFF FFHOM FFIELDC FFINTBAS \
     FFNBP FFNB FFNBX FFPOLY2 FFPOLY FFP \
     FFSLPE FF FFX FGLMICPK FGROUP FIELD \
     FILECAT FILE FINAALG FINITE FINRALG FLAGG2 \
     FLAGG FLALG FLASORT FLINEXP FLOATCP FLOATRP \
     FLOAT FMCAT FMC FMFUN FMONOID \
     FM FM2 FMTC FNAME FNCAT FNLA FOP \
     FORDER FORMULA1 FORMULA FORTCAT FORTFN FORTRAN \
     FORT FPARFRAC FPATMAB FPC FPS FR2 \
     FRAC2 FRAC FRAMALG FRETRCT FRIDEAL2 FRIDEAL \
     FRMOD FRNAAF2 FRNAALG FR FRUTIL FS2EXPXP \
     FS2 FS2UPS FSAGG2 FSAGG FSCINT FSERIES \
     FSINT FSPECF FSPRMELT FSRED FS FST \
     FSUPFACT FTEM FT FUNCTION FVC FVFUN \
     GALFACT GALFACTU GALPOLYU GAUSSFAC GBEUCLID \
     GBF GBINTERN GB GCDDOM GCNAALG GDMP \
     GENEEZ GENMFACT GENPGCD GENUFACT GENUPS GHENSEL \
     GMODPOL GOSPER GPOLSET GRALG GRAY GRDEF \
     GRIMAGE GRMOD GROEBSOL GROUP GSERIES GSTBL \
     GTSET HACKPI HASHTBL HB HDMP HDP \
     HEAP HELLFDIV HEUGCD HEXADEC HOAGG HYPCAT \
     IALGFACT IAN IARRAY1 IARRAY2 IBACHIN IBATOOL \
     IBITS IBPTOOLS ICARD ICDEN IDEAL IDECOMP \
     IDPAG IDPAM IDPC IDPOAM IDPOAMS IDPO \
     IEVALAB IFAMON IFARRAY IFF IIARRAY2 ILIST \
     IMATLIN IMATQF IMATRIX INBFF INCRMAPS INDE \
     INEP INFINITY INFORM1 INFORM INFPROD0 INFSP \
     INMODGCD INNMFACT INPRODFF INPRODPF INPSIGN INS \
     INTABL INTAF INTALG INTBIT INTCAT INTDOM \
     INTEF INTFACT INTG0 INTHEORY INTHERAL \
     INTHERTR INTPAF INTPM INTRAT INTRET \
     INTRF INTRVL INTSLPE INT INTTOOLS INTTR \
     INVLAPLA IPADIC IPF IPRNTPK IR2F IR2 \
     IROOT IRREDFFX IRRF2F IRSN IR IRURPK \
     ISTRING ISUMP ISUPS ITAYLOR ITFUN2 ITFUN3 \
     ITRIGMNP ITUPLE IVECTOR IXAGG JORDAN KAFILE \
     KDAGG KERNEL2 KERNEL KOERCE KONVERT KOVACIC \
     LALG LAPLACE LA LAUPOL LAZM3PK LEADCDET \
     LEXP LEXTRIPK LFCAT LF LGROBP LIB \
     LIECAT LIE LIMITPS LIMITRF LINDEP LINEXP \
     LIST2MAP LIST2 LIST3 LIST LMDICT LMODULE \
     LMOPS LNAGG LODEEF LODO1 LODO2 LODOCAT \
     LODOF LODOOPS LODO LOGIC LO LPEFRAC \
     LPOLY LSAGG LSMP1 LSMP LSPP LSQM \
     LWORD LZSTAGG M3D MAGMA MAPHACK1 MAPHACK2 \
     MAPHACK3 MAPPKG1 MAPPKG2 MAPPKG3 MATCAT2 MATCAT \
     MATLIN MATRIX MATSTOR MCALCFN MCDEN MCMPLX \
     MDAGG MDDFACT MESH MFINFACT MFLOAT MHROWRED \
     MINT MKBCFUNC MKCHSET MKFUNC MKRECORD \
     MKUCFUNC MLIFT MLO MMAP MODFIELD MODMONOM \
     MODMON MODOP MODRING MODULE MOEBIUS MONAD \
     MONADWU MONOGEN MONOID MONOTOOL MPC2 MPC3 \
     MPCPF MPOLY MPRFF MRATFAC MRCAT MRF2 MRING \
     MSETAGG MSET MSYSCMD MTHING MTSCAT MULTFACT \
     MULTSQFR NAALG NARNG \
     NASRING NCEP NCNTFRAC NCODIV NFINTBAS \
     NLINSOL NNI NODE1 NONE1 NONE NORMMA \
     NORMPK NORMRETR NPCOEF NREP NSMP NSUP2 \
     NSUP NTPOLFN NTSCAT NUMERIC NUMFMT \
     NUMODE NUMQUAD NUMTUBE OAGROUP OAMON OAMONS \
     OASGP OCAMON OC OCTCT2 OCT \
     ODECONST ODEEF ODEINT ODEPAL \
     ODEPRIM ODEPRRIC ODERAT ODERED ODERTRIC \
     ODESYS ODETOOLS ODPOL ODP ODR ODVAR \
     OFMONOID OINTDOM OMCONN OMDEV OMENC OMERRK \
     OMERR OMEXPR OMLO OMPKG OMSAGG OMSERVER \
     OM ONECOMP2 ONECOMP OPQUERY OP \
     ORDCOMP2 ORDCOMP ORDFIN ORDFUNS \
     ORDMON ORDRING ORDSET OREPCAT OREPCTO ORESUP \
     OREUP ORTHPOL OSI OUTFORM OUT OVAR \
     OWP PADEPAC PADE PADICCT PADICRC \
     PALETTE PAN2EXPR PARPC2 PARPCURV PARSC2 \
     PARSCURV PARSU2 PARSURF PARTPERM PATAB PATLRES \
     PATMAB PATMATCH PATRES2 PATRES PATTERN1 PATTERN2 \
     PATTERN PBWLB PCOMP PDECOMP \
     PDRING PENDTREE PERMAN PERMCAT PERMGRP \
     PERM PFBR PFBRU PFECAT PFOQ PFO \
     PFOTOOLS PFRPAC PFR PF PGCD PGE \
     PGROEB PICOERCE PID PINTERPA PINTERP PI \
     PLEQN PLOT1 PLOT3D PLOT PLOTTOOL PMASSFS \
     PMASS PMDOWN PMFS PMINS PMKERNEL PMLSAGG \
     PMPLCAT PMPREDFS PMPRED PMQFCAT PMSYM PMTOOLS \
     PNTHEORY POINT POLTOPOL POLUTIL POLY2 POLY2UP \
     POLYCATQ POLYCAT POLYLIFT POLYROOT POLY PPCURVE \
     PREASSOC PRIMARR2 PRIMARR PRIMCAT PRIMELT PRIMES \
     PRINT PRODUCT PRQAGG PR PRS PRTITION \
     PSCAT PSCURVE PSETCAT PSETPK PSEUDLIN PSQFR \
     PTCAT PTFUNC2 PTPACK PTRANFN PUSHVAR PWFFINTB \
     QALGSET2 QALGSET QCMPACK QEQUAT QFCAT2 QFCAT \
     QFORM QUAGG QUATCAT QUATCT2 QUAT QUEUE \
     RADCAT RADFF RADIX RADUTIL RANDSRC RATFACT \
     RATRET RCAGG RCFIELD RDEEF RDEEFS RDETR \
     RDETRS RDIST RDIV REAL0Q REAL0 REALSOLV \
     REAL RECLOS REDORDER REF REGSET REP1 \
     REP2 REPDB REP REPSQ RESLATC RESRING \
     RESULT RETRACT RETSOL RFDIST RFFACTOR RFFACT \
     RF RGCHAIN RIDIST RING RINTERP RMATCAT \
     RMATRIX RMCAT2 RMODULE RNG RNS ROIRC \
     ROMAN RPOLCAT RRCC RSDCMPK RSETCAT \
     RSETGCD RULECOLD RULESET RULE RURPK SAEFACT \
     SAERFFC SAE SAOS SCACHE SCPKG SDPOL \
     SDVAR SEG2 SEGBIND2 SEGBIND SEGCAT SEG \
     SEGXCAT SETAGG SETCAT SETMN SEXCAT \
     SEXOF SEX SFORT SFQCMPK SFRGCD SFRTCAT \
     SGCF SGROUP SHDP SHP SIGNEF SIGNRF \
     SIMPAN SINT SKAGG SMATCAT SMITH SMP \
     SMTS SNTSCAT SOLVEFOR SOLVERAD SOLVESER SOLVETRA \
     SORTPAK SPACE3 SPACEC SPECOUT SPFCAT SPLNODE \
     SPLTREE SQMATRIX SRAGG SRDCMPK SREGSET STACK \
     STAGG STBL STEP STINPROD STREAM1 STREAM2 \
     STREAM3 STREAM STRICAT STRING STRTBL STTAYLOR \
     STTFNC STTF SUBRESP SUBSPACE SUCH SULS \
     SUMFS SUMRF SUP2 SUPFRACF SUP SUPXS \
     SUTS SWITCH SYMBOL SYMFUNC SYMPOLY SYMS \
     SYMTAB SYSSOLP TABLBUMP TABLEAU TABLE TANEXP \
     TBAGG TBCMPPK TEMUTL TEX1 TEX TEXTFILE \
     TOOLSIGN TOPSP TRANFUN TREE TRIGCAT TRIGMNIP \
     TRIMAT TRMANIP TSETCAT TS TUBE TUBETOOL \
     TUPLE TWOFACT TYPE UDPO UDVO UFD \
     ULS2 ULSCAT ULSCCAT ULSCONS ULS UNIFACT \
     UNISEG2 UNISEG UP2 UPCDEN UPDECOMP UPDIVP \
     UPMP UPOLYC2 UPOLYC UPSCAT UP UPSQFREE \
     UPXS2 UPXSCAT UPXSCCA UPXSCONS UPXSSING UPXS \
     URAGG UTS2 UTSCAT UTSODE UTSODETL UTS \
     VARIABLE VECTCAT VECTOR2 VECTOR VIEW2D VIEW3D \
     VIEWDEF VIEW VOID VSPACE WEIER \
     WP WUTSET XALG XDPOLY XEXPPKG XFALG \
     XF XPOLYC XPOLY XPR XRPOLY \
     YSTREAM ZDSOLVE ZLINDEP ZMOD

# Full list of categories
CATLIST= AHYP ATTREG CFCAT ELTAB KOERCE KONVERT OM PRIMCAT PTRANFN \
        SPFCAT TYPE BASTYPE IEVALAB AGG EVALAB SETCAT HOAGG CLAGG \
        RADCAT RETRACT TRIGCAT ATRIG HYPCAT ELEMFUN ELTAGG IXAGG \
        LNAGG ORDSET FLAGG A1AGG ABELSG ABELMON CABMON ABELGRP \
        LMODULE RMODULE BMODULE MODULE SGROUP MONOID RNG RING ALGEBRA \
        COMRING ENTIRER DIVRING INTDOM GCDDOM PID EUCDOM UFD FIELD \
        ACF CHARNZ CHARZ ES LINEXP FLINEXP PATMAB FPATMAB FRETRCT \
        GROUP PDRING PATAB FS ACFS BGAGG DIOPS DIAGG ELAGG KDAGG \
        RCAGG URAGG STAGG LSAGG TBAGG ALAGG AMR ARR2CAT BRAGG LOGIC \
        BTAGG BTCAT CACHSET COMBOPC DIFRING DIFEXT FPC FINITE STEP \
        FFIELDC FINRALG FRAMALG FEVALAB MONOGEN PFECAT TRANFUN COMPCAT \
        OASGP OAMON OCAMON OAGROUP OAMONS ORDRING VSPACE DIRPCAT DLAGG \
        FAMR POLYCAT DPOLCAT QUAGG SKAGG DQAGG DVARCAT FAMONC XF FAXF \
        FDIVCAT FFCAT FILECAT MONAD NARNG NAALG FINAALG LIECAT FLALG \
        FORTCAT FMC FMCAT FMFUN FMTC FNCAT FORTFN REAL RNS FPS FRNAALG \
        SETAGG FSAGG FVC FVFUN GRMOD GRALG IDPC OINTDOM INS INTCAT LALG \
        LFCAT OREPCAT LODOCAT LZSTAGG MATCAT MDAGG MLO MONADWU MRCAT MSETAGG \
        PSCAT MTSCAT NASRING PSETCAT TSETCAT RSETCAT NTSCAT OC \
        PRQAGG OMSAGG ORDFIN ORDMON PADICCT PERMCAT \
        PPCURVE PSCURVE VECTCAT PTCAT QFCAT QUATCAT RCFIELD RMATCAT \
        RPOLCAT RRCC SEGCAT SEGXCAT SEXCAT SFRTCAT SMATCAT SNTSCAT \
        SPACEC SRAGG STRICAT UPSCAT ULSCAT ULSCCAT UPOLYC UPXSCAT \
        UPXSCCA UTSCAT XALG XFALG XPOLYC


# List of core domains
DOMLIST= ALIST BOOLEAN CHAR DFLOAT FRAC ILIST IARRAY1 ISTRING IVECTOR \
        INT LIST NNI OUTFORM PI PRIMARR REF SINT STRING SYMBOL VECTOR

# List of categories which provide default implementations
CATDOMS= A1AGG ABELGRP ABELMON ABELSG ACF ACFS AGG ALGEBRA AMR ARR2CAT \
        ATRIG BASTYPE BGAGG BRAGG BTAGG BTCAT CLAGG COMPCAT DIAGG \
        DIFEXT DIFRING DIOPS DIRPCAT DIVRING DPOLCAT DVARCAT ELAGG \
        ELEMFUN ELTAGG ES EUCDOM EVALAB FAMR FAXF FDIVCAT FEVALAB \
        FFCAT FFIELDC FIELD FINAALG FINRALG FLAGG FLINEXP FPC FPS \
        FRAMALG FRETRCT FRNAALG FSAGG FS GCDDOM GRALG GRMOD GROUP \
        HOAGG HYPCAT IEVALAB INS INTDOM IXAGG KDAGG LALG LIECAT LNAGG \
        LODOCAT LOGIC LSAGG LZSTAGG MATCAT MODULE MONAD MONADWU \
        MONOGEN MONOID NAALG NARNG NASRING OC ORDRING ORDSET OREPCAT \
        PDRING PFECAT POLYCAT PSCAT PSETCAT QFCAT QUATCAT RADCAT \
        RCAGG RCFIELD RETRACT RING RMATCAT RNS RPOLCAT RRCC RSETCAT \
        SETAGG SETCAT SGROUP SMATCAT SRAGG STAGG TBAGG TRANFUN \
        TRIGCAT TSETCAT UFD ULSCCAT UPOLYC UPSCAT UPXSCCA URAGG \
        UTSCAT VECTCAT VSPACE XF

EXTRASPADS=BPADICRT BPADIC COMMONOP GALUTIL MKFLCFN PADICRAT PADIC \
       SET WFFINTBS XPBWPOLY

BADSPADS=\
ALGFF BRAGG CHAR CMPLXRT COMPCAT COMPLEX COORDSYS \
DEGRED DFLOAT \
EF EXPR FAXF FC FFCGP FFF FFHOM FFIELDC \
FFNBP FFPOLY2 FFPOLY FPC FRIDEAL FRNAALG FR FSCINT FS GALFACT GENUPS \
GRIMAGE IARRAY1 ILIST INTEF INTPAF INTTOOLS IPF IRURPK ISUMP \
LODOF LZSTAGG MCMPLX MLIFT MULTSQFR NORMPK NUMODE ODERTRIC ODESYS \
OREPCTO PATRES PF PFO PMTOOLS PSEUDLIN RCAGG RDEEF RESRING RFDIST \
RPOLCAT RURPK SHP SOLVETRA SUPFRACF SYMFUNC TBCMPPK TEMUTL URAGG \
UTS WEIER WP ZDSOLVE ZMOD
@

<<layer0 bootstrap>>=
stamp-db: $(patsubst %, %.spad, ${SPADLIST} ${EXTRASPADS})
	if [ ! -f stamp-db ] ; then \
          grep -h '^)abbrev ' *.spad > abbrevs.input ; \
          cp ${srcdir}/boo* . ; \
          sed 's,@BBLD@,$(axiom_build_bindir),;s,@FASLEXT@,${FASLEXT},' ${srcdir}/boot-sum.in > boot-sum.sh ; \
          chmod 755 boot-sum.sh ; \
          mkdir -p r1/algebra ; \
          cp $(axiom_src_datadir)/algebra/*.daase r1/algebra ; \
          mkdir axp ; \
          cp -r ${AXIOM}/autoload axp ; \
          cp -r ${AXIOM}/algebra axp ; \
          cp -r ${AXIOM}/share axp ; \
          $(INSTALL_DATA) $(axiom_src_datadir)/algebra/gloss.text . ; \
          $(INSTALL_DATA) $(axiom_src_datadir)/algebra/topics.data  . ; \
          ./boot-sum.sh && touch stamp-db ; \
        fi

stamp-oboo3: stamp-db
	if [ ! -f stamp-oboo3 ] ; then \
           echo "Bootstrap object copy" ; \
           for A in ${CATLIST} ${DOMLIST} ; do \
              cp $$A.NRLIB/code.$(FASLEXT) ${OUT}/$$A.$(FASLEXT) || exit 1 ; \
           done; \
           for A in ${CATDOMS} ; do \
              cp $${A}-.NRLIB/code.$(FASLEXT) ${OUT}/$${A}-.$(FASLEXT)  || exit 1 ; \
           done; \
           rm -rf *.NRLIB \
           echo "Stage 3 object bootstrap (normal mode)" ; \
           echo > oboo3.input ; \
           for A in ${CATLIST} ${DOMLIST} ; do \
              echo ')compile "'$$A'.spad"' >> oboo3.input ; \
           done; \
           ( echo ')read "oboo3.input"' | \
              DAASE=./r7 ${INTERPSYS} ) || exit 1 ; \
	   echo "Stage 3 object copy" ; \
	   for A in ${CATLIST} ${DOMLIST} ; do \
	      cp $$A.NRLIB/code.$(FASLEXT) ${OUT}/$$A.$(FASLEXT) || exit 1 ; \
	   done ; \
	   for A in ${CATDOMS} ; do \
	      cp $${A}-.NRLIB/code.$(FASLEXT) ${OUT}/$${A}-.$(FASLEXT)  || exit 1 ; \
	   done ; \
	   touch stamp-oboo3 ; \
	fi

stamp-bootstrap: stamp-oboo3
	if [ ! -f stamp-bootstrap ] ; then \
	   for A in ${SPADLIST} ; do \
	      echo ')compile "'$$A'.spad"' | DAASE=./r7 ${INTERPSYS} ; \
	   done ; \
	   echo ')lisp (make-databases "" nil)' | DAASE=./r7 ${INTERPSYS} ; \
	   mkdir -p r8/algebra ; \
	   mv browse.daase  category.daase  compress.daase  interp.daase \
	      operation.daase r8/algebra || exit 1 ; \
           for A in ${EXTRASPADS} ; do \
              echo ')compile "'$$A'.spad"' ; \
           done | DAASE=./r8 ${INTERPSYS} ; \
           echo ')lisp (make-databases "" nil)' | DAASE=./r8 ${INTERPSYS} ; \
           mkdir -p rf/algebra ; \
           mv browse.daase  category.daase  compress.daase  interp.daase \
             operation.daase rf/algebra || exit 1 ; \
	   for A in ${BADSPADS} ; do \
	     echo ')compile "'$$A'.spad"' | DAASE=./rf ${INTERPSYS} ; \
           done ; \
	   for A in ${BADSPADS} ; do \
	     cp $$A.NRLIB/code.$(FASLEXT) ${OUT}/$$A.$(FASLEXT) || exit 1 ; \
	   done ; \
	   touch stamp-bootstrap ; \
	fi     

ifneq ($(BUILD_WAY),use_lisp)
$(patsubst %, ${OUT}/%.$(FASLEXT), ${SPADLIST}) : stamp-bootstrap
endif
@

\subsection{User Layer for newly added algebra}
Rather than classify newly created algebra into the existing type lattice
we add it here.
<<USERLAYER>>=

GUESSLIST= SMPEXPR FAMR2 NEWTON UFPS GOPT GUESSF1 \
           UTSSOL FFFG UFPS1 GOPT0 EXPRSOL FFFGF \
           RECOP GUESS GUESSINT GUESSP GUESSF GUESSAN

@

\section{Broken Files}
These files are Aldor files
\begin{verbatim}
axtimer.as Timer
iviews.as  InventorRenderPackage IVREND
ffrac.as   FormalFraction FORMAL
iviews.as  InventorViewPort IVVIEW
iviews.as  InventorDataSink IVDATA
herm.as    PackedHermitianSequence PACKED
nsfip.as   NagSpecialFunctionsInterfacePackage NAGSPE
nrc.as     NagResultChecks NAGRES
nqip.as    NagQuadratureInterfacePackage NAGQUA
noptip.as  NagOptimizationInterfacePackage NAGOPT
nepip.as   NagEigenInterfacePackage NAGEIG
ndftip.as  NagDiscreteFourierTransformInterfacePackage NAGDIS
\end{verbatim}

These domains are referenced but don't exist
\begin{verbatim}
OBJECT
\end{verbatim}

\section{The Environment}
\subsection{The working directories}
We define 5 directories for this build. The{\bf IN} directory
contains the pamphlet files for the algebra. These are expanded
into the{\bf MID} directory as either .spad or .as files. The
.spad files are compiled by the native spad internal compiler.
The .as files are compiled using the Aldor compiler. The output
of the compilation has two purposes. Part of the information is
used to build various database files (daase files). The other
part is executable code which is placed in the {\bf OUT}
directory. When invoked as ``make document'' we construct
the .dvi files in the{\bf DOC} directory.

The [[OUTSRC=$(axiom_target_srcdir)/algebra]] subdirectory contains the 
algebra source files extracted from the pamphlet files. These sources 
allow the end user to change the algebra if needed.

<<environment>>=

OUT=$(axiom_targetdir)/algebra
DOC=$(axiom_target_docdir)/src/algebra
OUTSRC=$(axiom_target_srcdir)/algebra
# INPUT=${INT}/input
INPUT=../input

@

\subsection{The interpsys variable}
The {\bf interpsys} image is the compile-time environment for algebra
files.

<<environment>>=

INTERPSYS=AXIOM=${AXIOM} $(axiom_build_bindir)/interpsys

@
\subsection{The spad source list}
Note that we have excluded {\bf mlift.spad.jhd} from this list.
We need to figure out which mlift.spad to keep.

<<environment>>=

SPAD_SRCS= \
     acplot aggcat2 aggcat algcat algext algfact \
     algfunc allfact alql any array1 \
     array2 attreg bags bezout boolean \
     brill card carten \
     catdef cden clifford clip cmplxrt coerce \
     color combfunc combinat complet constant contfrac \
     coordsys cra crfp curve cycles \
     cyclotom ddfact defaults defintef \
     defintrf degred derham dhmatrix divisor dpolcat \
     drawopt drawpak draw efstruc efuls efupxs \
     eigen elemntry elfuts equation1 equation2 error \
     expexpan expr2ups exprode expr \
     facutil ffcat ffcg fffg \
     fff ffhom ffnb ffpoly2 ffpoly ffp \
     ffx files float fmod fname fnla \
     formula fortcat fortmac fortpak fortran forttyp \
     fourier fparfrac fraction free fr fs2expxp \
     fs2ups fspace funcpkgs galfact galfactu \
     galpolyu galutil gaussfac gaussian gbeuclid gbintern \
     gb gdirprod gdpoly geneez generic genufact \
     genups ghensel gpgcd gpol grdef groebf \
     groebsol gseries ideal idecomp indexedp infprod \
     intaf intalg intaux intclos intef integer \
     integrat interval intfact intpm intrf irexpand \
     irsn ituple kl kovacic laplace laurent \
     leadcdet lie limitps lindep lingrob liouv \
     listgcd list lmdict lodof lodop lodo \
     manip mantepse mappkg matcat matfuns matrix \
     matstor mesh mfinfact misc mkfunc mkrecord \
     mlift moddfact modgcd modmonom modmon modring \
     moebius mring mset mts multfact multpoly \
     multsqfr naalgc naalg newdata newpoint newpoly \
     nlinsol nlode npcoef nregset nsregset numeigen \
     numeric numode numquad numsolve numtheor oct \
     odealg odeef oderf omcat omdev omerror \
     omserver opalg openmath op ore outform \
     out pade padiclib padic paramete partperm \
     patmatch1 patmatch2 pattern pcurve pdecomp perman \
     permgrps perm pfbr pfo pfr pf \
     pgcd pgrobner pinterp pleqn plot3d plot \
     plottool polset poltopol polycat poly primelt \
     print product prs prtition pscat pseudolin \
     ptranfn puiseux qalgset quat radeigen radix \
     random ratfact rdeef rderf rdesys real0q \
     realzero reclos rec regset rep1 rep2 \
     resring retract rf riccati rinterp \
     rule seg setorder sets sex sf \
     sgcf sign si smith solvedio solvefor \
     solvelin solverad sortpak space special sregset \
     ssolve stream string sttaylor sttf \
     sturm suchthat suls sum sups supxs \
     suts symbol syssolp system tableau table \
     taylor tex transsolve tree trigcat \
     triset tube twofact unifact updecomp updivp \
     utsode variable vector view2D view3D viewDef \
     viewpack void weier wtpol xlpoly xpoly \
     ystream zerodim

SPADFILES=$(patsubst %, $(OUTSRC)/%.spad.pamphlet, $(SPAD_SRCS))

@
\subsection{The ALDORFILES list}
<<environment>>=

ALDOR_SRCS=\
  axtimer ffrac herm interval invnode invrender \
  invtypes invutils iviews ndftip nepip noptip \
  nqip nrc nsfip

ALDORFILES=$(patsubst %, $(OUTSRC)/%.as.pamphlet, $(ALDOR_SRCS))

@

\subsection{The pamphlets list}
<<environment>>=

pamphlets=$(addsuffix .as.pamphlet, $(ALDOR_SRCS)) \
          $(addsuffix .spad.pamphlet, $(SPAD_SRCS)) \
	  Makefile.pamphlet exposed.lsp.pamphlet

@

\section{Test Cases}
<<environment>>=

TESTS=${INPUT}/INTHEORY.input ${INPUT}/VIEW2D.input ${INPUT}/TESTFR.input

@
<<testrules>>=

${INPUT}/TESTFR.input: $(srcdir)/fr.spad.pamphlet
	$(axiom_build_document) --tangle='TEST FR' --output=$@ $<

${INPUT}/INTHEORY.input: $(srcdir)/numtheor.spad.pamphlet
	$(axiom_build_document) --tangle='TEST INTHEORY' --output=$@ $<

${INPUT}/VIEW2D.input: $(srcdir)/view2D.spad.pamphlet
	$(axiom_build_document) --tangle='TEST VIEW2D' --output=$@ $<

@

\section{Generic Make Rules}

The idea is to use generic rules to try to cut down the size of this file.

This Makefile works very hard to cache
intermediate results in order to minimize the re-build time. The cached
files are kept in the [[${MID}]] directory. If one of these
files disappears but the original pamphlet file is unchanged we only
need to rebuild the final file. These rule will attempt to do
that and they succeed however these are intermediate files created by
implicit rules so they would normally be deleted. To prevent the removal
the NRLIB directory and its contents, the files are marked as .PRECIOUS.
Unfortunatly, this means that we have to explicitly delete NRLIB
directory when it gets out of date.

The output of the compile step is saved in a file of the same name
and extension [[.$(FASLEXT)]] in the [[${OUT}]] directory, so we need
to copy it from [[${MID}]] directory.

<<genericDotOfiles>>=

ifneq ($(BUILD_WAY),use_lisp)
${OUT}/%.$(FASLEXT): %.NRLIB/code.$(FASLEXT)
	cp $*.NRLIB/code.$(FASLEXT) $@
endif

@ 
<<genericNRLIBfiles>>=

.PRECIOUS: %.NRLIB/code.$(FASLEXT)
%.NRLIB/code.$(FASLEXT): %.spad
	@ echo compiling $*.spad to $*.NRLIB
	rm -rf $*.NRLIB
	( \
	     echo ')co $*.spad' | DAASE=./rf ${INTERPSYS} \
	   )

@

<<genericSPADfiles>>=

$(OUTSRC)/%.spad.pamphlet: $(srcdir)/%.spad.pamphlet
	$(INSTALL_DATA) $< $@

${OUTSRC}/%.spad: $(srcdir)/%.spad.pamphlet
	@ echo tangling $< to $@
	@ $(axiom_build_document) --tangle --output=$@ $<

@

<<genericRules>>=

<<genericDotOfiles>>
<<genericNRLIBfiles>>
<<genericSPADfiles>>

@

\section{Pamphlet file structure}
Because the individual .spad files are grouped into higher-level
algebra pamphlet files, the rules for extracting them are coded
below as simple ``awk'' scripts that are called when the Makefile
is constructed.

<<findAlgebraFiles>>=

<<findSpadFiles>>
@

There are, at present, 2 kinds of algebra files to be handled.
First we have [[.as]] files which use the [[aldor]] compiler.
These are ignored here as the compiler is not yet integrated.

Second, there are 3 "types" of algebra which are all treated 
the same at compile time, namely the "domain", "category", and
"package" algebra.

\subsection{Finding the algebra code}
Step 1 is to scan all of the algebra pamphlet files for the
chunk names which contain the string "domain", "package", or
"category". This is done using egrep (same as grep -E, which
means that the pattern is an extended regular expression) because
extended regular expressions allows the use of alternatives
written as (domain|package|category). Thus the command
\begin{verbatim}
 egrep '@<<(domain|package|category) .*>>=' *.spad.pamphlet 
\end{verbatim}
will scan the algebra files looking for special chunknames.
Axiom's chunk names are written in a stylized form so that each
algebra chunk name begins with one of those three symbols. Thus 
in zerodim.spad.pamphlet the LexTriangularPackage chunkname is:
\begin{verbatim}
@<<package LEXTRIPK LexTriangularPackage>>
\end{verbatim}
so this egrep will generate an output line, prefixed by the filename
that looks like:
\begin{verbatim}
zerodim.spad.pamphlet:@<<package LEXTRIPK LexTriangularPackage>>=
\end{verbatim}
There can be many lines of output per pamphlet file, one for
each domain, package and category code chunk contained in the file.

Step 2 is an [[awk]] command line.

\subsection{Write the Makefile stanzas for the algebra files}

<<findSpadFiles>>=

egrep '@<<(domain|package|category) .*>>=' *.spad.pamphlet | sort | uniq | \
awk -F: '{
  chunk=substr($2,3,length($2)-5);
  split(chunk,part," ");
  spadfile=part[2]".spad";
  print spadfile": $(srcdir)/"$1;
  print "\t@${TANGLE} -R\""chunk"\" $(srcdir)/"$1">"spadfile;
  print "";
}'

@
[awk] processes each line of the [[egrep]] output. 

The awk script uses [[-F:]] which is a flag that says that a [[:]] is
the field separator. As a result the \$1 and \$2 in the awk script
refer to the parts of the egrep output that come before and after the
[[:]] respectively.

The variable [[chunk]] is assigned the actual chunk name minus
the @<< and >>= delimiters. In the example given above this will become
\begin{verbatim}
package LEXTRIPK LexTriangularPackage
\end{verbatim}
The call to [[split]] splits the chunk into parts separated
by spaces. Thus
\begin{verbatim}
  part[1]=package
  part[2]=LEXTRIPK
  part[3]=LexTriangularPackage
\end{verbatim}
The variable [[spadfile]] in the above example is set to
\begin{verbatim}
${MID}/LEXTRIPK.spad
\end{verbatim}
Finally, in the domain example given above we print two lines.
The first line is the Makefile stanza header which depends on the
original [[zerodim.spad.pamphlet]] file.

The second line is the body of the makefile stanza which calls 
notangle to extract the algebra from the original pamphlet using
the chunk name and writes it to the intermediate subdirectory. In
the case above this would resolve to [[\${MID}/LEXTRIPK.spad]].

For the line given above it outputs the following:
\begin{verbatim}
${MID}/LEXTRIPK.spad: $(srcdir)/zerodim.spad.pamphlet
	$(axiom_build_document) --tangle='package LEXTRIPK LexTriangularPackage' --output=$@ $<
\end{verbatim}


\section{The Makefile}

<<*>>=

<<environment>>

subdir = src/algebra/


<<bootstrap variables>>
<<USERLAYER>>

# The algebra build is not yet ready for parallel build.
.NOTPARALLEL:

.PHONY: all all-algebra
all: all-ax

all-ax all-algebra:
	if [ -f use_lisp ] ; then \
          ${MAKE} BUILD_WAY=use_lisp stamp-alg ; \
	else \
	  ${MAKE} BUILD_WAY=normal stamp ; \
	fi
	@ echo finished $(builddir)

ifeq ($(BUILD_WAY),use_lisp)
FULL_FASL_LIST=${SPADLIST} ${EXTRASPADS} ${GUESSLIST} \
               $(patsubst %, %-, ${CATDOMS})

stamp-alg: stamp-fasls ${TESTS}

stamp-fasls: $(patsubst %, ${OUT}/%.$(FASLEXT), ${FULL_FASL_LIST})

$(patsubst %, ${OUT}/%.$(FASLEXT), ${FULL_FASL_LIST}) : \
	${OUT}/%.$(FASLEXT) : %.$(FASLEXT)
	cp $< $@

.PRECIOUS: %.$(FASLEXT)

$(patsubst %, %.$(FASLEXT), ${FULL_FASL_LIST}) : \
        %.$(FASLEXT) : %.lsp
	echo ')lisp (progn (compile-file "$<") (quit))' | \
          DAASE=$(axiom_src_datadir) $(INTERPSYS)

.PRECIOUS: %.lsp
$(patsubst %, %.lsp, ${FULL_FASL_LIST}) : \
	%.lsp : $(srcdir)/%.lsp
	cp $< $@
endif

<<layer0 bootstrap>>

guess-pkg: $(patsubst %, %.spad, ${GUESSLIST}) stamp-src
	( \
	  { for A in ${GUESSLIST}; do \
	        echo ')compile "'$$A'.spad"' ; \
	    done | DAASE=./rf ${INTERPSYS} ; \
	    for A in ${GUESSLIST}; do \
	        cp $$A.NRLIB/code.$(FASLEXT) ${OUT}/$$A.$(FASLEXT) || exit 1 ; \
	    done ; \
          } \
	)
	touch guess-pkg

stamp: stamp-src guess-pkg ${TESTS}
	-rm -f stamp
	$(STAMP) stamp

everything: check lib db cmd gloss
	@ echo 4303 invoking make in `pwd` with parms:
	@ echo LSP= ${LSP} PART= ${PART} SUBPART= ${SUBPART}
	@ echo SRC= ${SRC} INT= ${INT}
	@ echo LISP=${LISP} BYE=${BYE}

stamp-src: stamp-spadsrc stamp-bootstrap \
             $(patsubst %, ${OUT}/%.$(FASLEXT), ${SPADLIST} ${EXTRASPADS})
	if [ ! -f stamp-src ] ; then \
	    echo 4304 Done building NRLIBS from spad sources ; \
	    touch stamp-src ; \
	fi


stamp-spadsrc: ${SPADFILES}
	touch stamp-spadsrc

check:
	@ echo 4305 Checking that INTERP.EXPOSED and NRLIBs are consistent
	@ echo 4306 libcheck needs to use exposed.lsp, not INTERP.EXPOSED


<<genericRules>>

<<testrules>>


mostlyclean-local: ;

clean-local: mostlyclean-local

distclean-local: clean-local

include extract-spad.mk

.NOTPARALLEL:

@
\eject
\begin{thebibliography}{99}
\bibitem{1} nothing
\end{thebibliography}
\end{document}
