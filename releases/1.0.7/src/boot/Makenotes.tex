%% Oh Emacs, this is a -*- Makefile -*-, so give me tabs.
\documentclass{article}
\usepackage{axiom}

\title{\File{src/boot/Makefile} Pamphlet}
\author{Timothy Daly \and Gabriel Dos~Reis}

\begin{document}
\maketitle

\begin{abstract}
  \Tool{Axiom} is built in layers. The first layer is contructed into
  an image called {\bf bootsys}. The \Tool{bootsys} image is used
  to translate Boot code to Common Lisp code.  Since a Boot coded
  interpreter is needed to translate the code for the Boot coded
  interpreter we have a ``boot-strapping'' problem.  In order to get
  the whole process to start we need certain files kept in 
  common lisp form. This directory contains those files.
\end{abstract}
\eject

\tableofcontents
\eject

\section{Introduction}
\label{sec:intro}
 
The Scratchpad language is implemented by using a mixture of Lisp and
a more convenient language for writing Lisp called \emph{Boot}.
This document contains a description of the Boot language, and some
details of the resulting Lisp programs.
The description of the translation
functions available are at the end of this file.
 
The main difference between Lisp and Boot is in the syntax for
the application of a function to its argument.
The Lisp format [[(F X Y Z)]], means, when [[F]] is a function,
the application of [[F]] to its arguments [[X]], [[Y]], and [[Z]],
is written in Boot as [[F(X,Y,Z)]].
When [[F]] is a special Lisp word it will be written
in Boot by using some other syntactic construction, such as spelling
in CAPITAL LETTERS.
 
Boot contains an easy method of writing expressions that denote lists,
and provides an analogous method of writing patterns containing variables
and constants which denote a particular class of lists.  The pattern
is matched against a particular list at run time,
and if the list belongs to the class then its variables will
take on the values of components of the list.  Similarly, Boot provides
an easy way of writting discriminated unions or algebraic types, and 
pattern matching as found in ML.
 
 A second convenient feature provided by Boot is a method of
writing programs that iterate over the elements of one or more lists
and which either transform the state of the machine, or
produce some object from the list or lists.


\section{Boot To Common Lisp Translaters}
\label{sec:boot-to-cl}

The Boot to Common Lisp translation is organized in several 
separate logical phases.  At the moment, those phases are not
really separate; but from a logical point of view, it is better
to think of them that way.


\subsection{The Boot Includer}
\label{sec:boot-to-cl:includer}

The Boot Includer is the module that reads Boot codes from source files.
The details of the Includer, as well as the grammar of the include
files are to be found in \File{btscan2.boot}


\subsection{The Scanner}
\label{sec:boot-to-cl:scanner}

The tokenization process is implemented in \File{btscan2.boot}.  Further 
details about keywords and reserved identifiers are available in
\File{typrops.boot}.


\subsection{Piling}
\label{sec:boot-to-cl:piling}

The Boot language uses layout to delimit blocks of expressions. After
the scanner pass, and before the parser pass is another pass called
\emph{piling}.  The piling pass inserts tokens to unambiguously delimit
the boundaries of piles.  This is implemented in 
\File{btpile2.boot}


\subsection{The Parser}
\label{sec:boot-to-cl:piling}

The Boot parser is implemented in \File{typars.boot}.  It is a hand-written 
recursive descent parser 
based on \emph{parser combinators} methodology.  Thoe files also
implicitly defines the grammar of the Boot language.


\subsection{The Transformer}
\label{sec:boot-to-cl:transfo}

As observed earlier, the Boot language was originally defined as a syntactic
sugar over Common Lisp.  Consequently, it semantics is defined by
tranformation to Lisp.  The transformers are defined in
\File{tytree1.boot}.  

\subsection{Utils}
\label{sec:boot-to-cl:utils}

Finally, the file \File{ptyout.boot} is a pot-pourri of many utility
functions.  It also contains the entry points to the Boot translater.


\section{Boot}
\label{sec:boot}
 
\subsection{Lines and Commands}
 
If the first character of a line is a closing parenthesis the line
is treated as a command which controls the lines that will be
passed to the translater rather than being passed itself.
The command [[)include filename]] filemodifier will for example
be replaced by the lines in the file [[filename filemodifier]].
 
If a line starts with a closing parenthesis it will be called a command
line, otherwise it will be called a plain line.
The command lines are
\begin{verbatim} 
name            as written
 
Include         )include filename filemodifier
IncludeLisp     )includelisp filename filemodifier
If              )if bootexpression
Else            )else
ElseIf          )elseif bootexpression
EndIf           )endif
Fin             )fin
Say             )say  string
Eval            )eval bootexpression
EvalStrings     )evalstrings bootexpression
Package         )package packagename
 
SimpleLine::= PlainLine | Include | IncludeLisp |Say | Eval | EvalStrings
              | Package
\end{verbatim} 

A [[PlainLine]] is delivered to the translater as is.
 
An [[Include]] delivers the lines in the file filename.filemodifier,
treated as boot lines.
 
An [[IncludeLisp]] delivers the lines in the specified file, treated as Lisp
lines. The only comments allowed in lisp files that are included in
this way require that the semicolon is at the beginning of the line.
 
A [[Say]] outputs the remainder of the line to the console,
   delivering nothing to the translater.
 
An [[Eval]] translates the reminder of the line, assumed to be
   written in Boot, to Lisp, and evaluates it, delivering nothing to
   the translater.
 
An [[EvalStrings]] also translates and evaluates the rest of the line
   but this time assumes that the Boot expression denotes a list
   of strings which are then delivered to the translater
   instead of the EvalString line. The strings are treated as Boot lines.
 
It is also possible to include or exclude lines based upon some
condition which is the result of translating and evaluating
the boot expression that follows an )if or )elseif command.
This construction will be called a Conditional. A file will be
composed from SimpleLines and Conditionals. A file is either
terminated by the end of file or by a Fin line.
\begin{verbatim}  
Components ::=(SimpleLine | Conditional)*
 
File ::= Components  ( Fin | empty)
 
A conditional is bracketed by an If and an EndIf.
 
Conditional ::= If Components Elselines EndIf
\end{verbatim} 

If the boot expression following the )if has value true then the
Components are delivered but not the ElseLines,
otherwise the Components are ignored ,and the ElseLines
are delivered to the translater. In any case the lines after
the EndIf are then processed.
\begin{verbatim}  
ElseLines ::= Else Components | ElseIf Components ElseLines | empty
\end{verbatim} 

When the Elselines of a Conditional is being included then if an
"Else Components" phrase is encountered then the following
Components are included
otherwise if an "ElseIf Components ElseLines" phrase is encountered then
the boot expression following the )elseif is evaluated and
if true the following Components are included, if false the
following ElseLines is included.
 
 
\subsection{Boot syntax and semantics}

The semantics of Boot was originally defined by translation to Lisp.
Ideally, we would like to give it a self-contained semantics, 
without explicitly referring to Lisp, or if we must we should use
lambda calculus.

\subsubsection{Source character set}
\label{sec:boot:char-set}

???What is the source character set???  That of Common Lisp?

\subsubsection{Identifiers}
\label{sec:boot:identifier}
 
The standard identifiers start with a letter ([[a-z]] or [[A-Z]])
dollar sign ([[$]]), question mark ([[?]]), or the percent sign 
([[\%]]), and are followed by any number of letters, digits, single 
quotes([[']]), question marks, or percent signs.
It is possible however, by using the escape character ([[\_]]), 
to construct identifiers that contain any
characters except the blank or newline character. The rules in this case
are that an escape character followed by any non-blank character
will start an identifier with that character.  Once an identifier
has been started either in this way or by a letter, [[$]], or 
[[%]], then it may be continued either with a letter, digit, 
quote , question mark or percent sign, or with
an escape character followed by any non-blank character.
Certain words having the form of identifiers are not classified as
such, but are reserved words. They are listed below.
 
An identifier ends when a blank or end of line is encountered, or
an escape character followed by a blank or end of line, or a
character which is not a letter, digit, quote, question mark
or percent sign is found. Two identifiers are equal if the
strings produced by replacing each escape followed by a character
by that character are equal character by character.
 
\subsubsection{Numbers}
\label{sec:boot:number}
 
Integers start with a digit ([[0-9]]) and are followed by any number
of digits.  The syntax for floating point numbers is
\begin{verbatim}
<.I | I. | I.I> <E|e> <+ | - | empty> I 
\end{verbatim}
where I is an integer.
 
\subsubsection{Strings}
\label{sec:boot:string}
 
Strings of characters are enclosed by double quote signs. They cannot
span two or more lines and an escape character within a string will
include the next character regardless of its nature.
The meaning of a string depends somewhat on the context in which
it is found, but in general a bare string denotes the interned atom
making up its body whereas when it is preceded by a single quote (')
it denotes the string of characters enclosed.
 
\subsubsection{S-expressions}
\label{sec:boot:s-expression}
 
An s-expression is preceded by a single quote and is followed by
a Lisp s-expression.
\begin{verbatim}  
sexpression ::=identifier | integer | MINUS integer | float | string
            | QUOTE sexpression | parenthesized sexpression1
 
sexpression1 ::=sexpression (DOT sexpression | sexpression1)| empty
\end{verbatim}  

There are two ways to quote an iddentifier: either 'name or "name", which
both give rise to (QUOTE name). However a string that is a
component of an sexpression will denote the string unless it is the
sole component of the s-expression in which case it denotes a string
i.e. '"name" gives rise to "name" in Lisp rather than (QUOTE "name").
 
 
\subsubsection{Keywords}
\label{sec:boot:keyword}
 
The table of key words follows, each is given an upper case
name for use in the description of the syntax.
\begin{verbatim} 
        as written      name
 
            and          AND
            by           BY
            case         CASE
            cross        CROSS
            else         ELSE
            for          FOR
            if           IF
            in           IN
            is           IS
            isnt         ISNT
            of           OF
            or           OR
            repeat       REPEAT
            return       RETURN
            structure    STRUCTURE
            then         THEN
            until        UNTIL
            where        WHERE
            while        WHILE
            .            DOT
            :            COLON
            ,            COMMA
            ;            SEMICOLON
            *            TIMES
            **           POWER
            /            SLASH
            +            PLUS
            -            MINUS
            <            LT
            >            GT
            <=           LE
            >=           GE
            =            SHOEEQ
            ^            NOT
            ^=           NE
            ..           SEG
            #            LENGTH
            =>           EXIT
            :=           BEC
            ==           DEF
            ==>          MDEF
            (            OPAREN
            )            CPAREN
            (|           OBRACK
            |)           CBRACK
            [            OBRACK
            ]            CBRACK
            suchthat     BAR
            '            QUOTE
            |            BAR
\end{verbatim}  
 
\subsubsection{Primary}
\label{sec:boot:primar-expr}

\begin{verbatim} 
constant::= integer | string | float | sexpression
\end{verbatim} 

The value of a constant does not depend on the context in which it
is found.
\begin{verbatim}  
primary::= name | constant | construct | block | tuple |  pile
\end{verbatim} 

The primaries are the simplest constituents of the language and
either denote some object or perform some transformation of the
machine state, or both.
The statements are the largest constituents and enclosing them
in parentheses converts them into a primary.
 
An alternative method of grouping uses indentation to indicate the
parenthetical structure.
A number of lines whose first non-space characters are in the same
column will be called a \emph{pile}.  The translater first tokenizes the
lines producing identifier, key word, integer, string or float tokens,
and then examines the pile structure of a Boot program
in order to add additional tokens called [[SETTAB]], [[BACKTAB]] 
and [[BACKSET]].
These tokens may be considered as commands for creating a pile.
The [[SETTAB]] starts a new line indented from the previous line and
pushes the resulting column number on to a stack of tab positions.
The [[BACKTAB]] will start a new line at the column position found
at the head of the stack and removes it from the stack.
The [[BACKSET]] has the same effect as a [[BACKTAB]] immediately followed
by a [[SETTAB]].
The meaning of a sequence of tokens containing [[SETTAB]],
[[BACKTAB]], and [[BACKSET]] is the same the sequence in which each
[[SETTAB]] is replaced by [[OPAREN]] , each [[BACKTAB]] is replaced by
[[CPAREN]], and each [[BACKSET]] is replaced by [[SEMICOLON]]. By
construction the [[BACKTABS]] and [[SETTABS]] are properly nested.
\begin{verbatim}  
listof(p,s)== p | p s ... s p
 
parenthesized s ::=   OPAREN s CPAREN
piled         s ::=   SETTAB s BACKTAB
 
blockof s ::=    parenthesized (listof (s,SEMICOLON))
pileof s  ::=    piled         (listof (s,BACKSET  ))
\end{verbatim} 

A pileof s has the same meaning as a blockof s.
There is however a slight difference because piling is weaker than
separation by semicolons. In other words the pile items
may be listof(s,SEMICOLON).
In other words if statements::= listof(statement,SEMICOLON) then
we can have a pileof statements which has the same meaning as
the flattened sequence formed by replacing
all [[BACKSET]]'s by [[SEMICOLON]]'s.
 
A blockof statement is translated to a compound statement
e.g. in the absence of any exits,
(a;b;c;d) is translated to (PROGN a b c d).
 
\subsubsection{Selectors}
\label{sec:boot:selector}

\begin{verbatim}  
selector::= leftassociative(primary, DOT)
\end{verbatim} 

A selector [[a.b]] denotes some component of a structure, and in
general is translated to [[(ELT a b)]]. There are some special identifiers
that may be used in the [[b]] position to denote list components, of which
more later.
The [[DOT]] has a greater precedence than juxtaposition and is
left associative, For example
\begin{verbatim} 
a.b.c  is grouped as (a.b).c which is translated to
  (ELT (ELT a b) c)
 
application ::= selector selector ... selector
 
\end{verbatim} 

Application of function to argument is denoted by juxtaposition.
 
A sequence of selectors is right associative and so
[[f g h x]] is grouped as [[f(g(h x))]]. The applications [[f x]] and 
[[f(x)]]
mean the application of [[f]] to [[x]] and get translated to
the Lisp [[(f x)]]. The application of a function to the empty list
is written [[f()]], meaning the Lisp [[(f)]].  [[f(x,y,z)]] gets translated to
the Lisp [[(f x y z)]].
Common Lisp does not permit a variable to occur in operator position,
so that when f is a variable its application has to be
put in argument position of a [[FUNCALL]] or [[APPLY]].
[[f(x,y,z)]] has to be replaced by [[FUNCALL(f,x,y)]] which gets translated to
the Lisp [[(FUNCALL f x y z)]].
In Common Lisp each symbol might refer
to two objects a function and a non-function. In order to resolve
this ambiguity when a function symbol appears in a context other
than operator position it has to be preceded by the symbol [[FUNCTION]].
Also it is possible to produce the function type symbol from the
non-function symbol by applying [[SYMBOL-FUNCTION]] to it.
 
Certain reserved words called infixed operators namely
[[POWER]], [[TIMES]], [[SLASH]], [[PLUS]], [[MINUS]], [[IS]],
[[EQ]], [[NE]] , [[GT]], [[GE]], [[LT]], [[LE]], [[IN]], [[AND]], 
[[OR]], indicate application by being placed between their 2 arguments.
 
Infixed application may be either right- or left-associative.
\begin{verbatim}  
rightassociative(p,o)::= p o p o p o ... o p
                    ==  p o (p o (p o ... o p)))
 
leftassociative(p,o)::= p o p o p o ... o p
                    ==  (((p o p) o p) o ...) o p
 
 
exponent ::= rightassociative(application,POWER)
 
reduction ::= (infixedoperator |string | thetaname) SLASH application
\end{verbatim} 

In a reduction the application denotes a list of items and
operator [[SLASH]] application accumulates the list elements from the
left using the operator
\begin{verbatim} 
e.g.  +/[a,b,c] means (((0+a)+b)+c)
\end{verbatim} 

Only certain operators are provided with values when the list is empty
they are [[and]], [[or]], [[+]], [[*]], [[max]], [[min]], [[append]], 
[[union]]. However any function can be used as an operator by enclosing it 
in double quotes. In this case the reduction is not applicable to an
empty list.
\begin{verbatim}  
multiplication ::= rightassociative(exponent,TIMES|SLASH) | reduction
 
minus ::= MINUS multiplication | multiplication
 
arith ::= leftasscociative(minus,PLUS | MINUS)
 
is ::= arith | arith (IS | ISNT) pattern
 
comparison ::= is (EQ | NE | GT | GE | LT | LE | IN) is | is
 
and  ::= leftassociative (comparison,AND)
 
return ::= and | RETURN and
 
expression ::= leftassociative(return,OR)
\end{verbatim}  

The infixed operators denote application of the function to its
two arguments. To summarize,
the infixed operators are, in order of decreasing precedence
strengths.
\begin{verbatim}  
        .
        juxtaposition
        **
        * /
        + -
        is
        = ^= > >= < <= in
        and
        or
\end{verbatim} 

\subsubsection{Conditionals}
\label{sec:boot:conditional}

\begin{verbatim}  
conditional ::= IF where THEN where |
                IF where THEN where ELSE where
 
IF a THEN b is translated to (COND (a b)) and
IF a THEN b else c is translated to (COND (a b) (T c))
 
statement::= conditional | loop | expression
\end{verbatim} 

\subsubsection{Loops}
\label{sec:boot:iteration}

\begin{verbatim} 
loop ::= crossproduct REPEAT statement | REPEAT statement
 
iterator ::= forin | suchthat | until | while
 
iterators ::= iterator iterator ... iterator
 
crossproduct ::=rightassociative(iterators,CROSS)
 
suchthat ::= BAR where
 
while ::= WHILE expression
 
until ::= UNTIL expression
 
forin ::= for variable IN segment |
          for variable IN segment BY arith
 
segment::= arith | arith SEG arith | arith SEG
\end{verbatim} 

A loop performs an iterated transformation of the state which is
specified by its statement component and its iterators.
The forin construction introduces a new variable which is assigned
the elements of the list which is the value of the segment in the order
in which they appear in the list .
 
A segment of the form [[arith]] denotes a list,
and segments of the form [[arith SEG arith]] and
[[arith SEG]] denote terminating and non-terminating
arithmetic progressions.
The [[BY arith]] option is the step size, if omitted the step is [[1]].
 
Two or more [[forin]]'s may control a loop.
The associated lists are scanned in parallel and
a variable of one [[forin]] may not appear in the segment expression that
denotes the list in a second [[forin]].
Such a variable may however occur in the conditions for filtering or
introduced by a [[suchthat]], or for termination introduced by a
while iterator, and in the statement of the loop.
The [[forin]] variables are local to the statement, the conditions
that follow a [[while]] or [[suchthat]] in the same list of iterators and
have no meaning outside them.
The loop will be terminated when one of its [[forin]] lists is null, or
if the condition in a [[while]] is not satisfied. The list
elements are filtered by all the [[suchthat]] conditions.
The ordering of the iterators is irrelevant to the meaning, so it is
best to avoid side effects within the conditions for filtering and
termination.
 
It is possible to control a loop by using a \emph{cross-product} of iterators.
The iteration in the case [[iterators1 CROSS iterators2]] is over
all pairs of list items one from the list denoted by
iterators1 and the other from the list denoted by iterators2.
In this case the variables introduced [[forin]] statements in 
[[iterators1]] may be used in [[iterators2]].
 
\subsubsection{Lists}
\label{sec:boot:list}
 
Boot contains a simple way of specifying lists that are constructed
by [[CONS]] and [[APPEND]], or by transforming one list to another in a
systematic manner.
\begin{verbatim}  
construct ::= OBRACK construction CBRACK
 
construction ::= comma | comma iteratortail
 
iteratortail ::= REPEAT iterators | iterators
\end{verbatim} 

A construct expression denotes a list and may also have a list
of controlling iterators having the same syntax as a loop. In this
case the expression is enclosed in brackets and the iterators follow
the expression they qualify, rather than preceding it.
 
In the case that there are no iterators the construct expression
denotes a list by listing its components separated by commas, or by
a comma followed by a colon. In the simple case in which there are no
colons the Boot expression [a,b,c,d] translates to the Lisp
[[(LIST a b c d)]] or [[(CONS a (CONS b (CONS c (CONS d NIL))))]].
 
When elements are separated by comma colon, however, the expression
that follows will be assumed to denote a list which will be appended
to the following list, rather than consed. An exception to this rule
is that a colon preceding the last expression is translated to
the expression itself. If it immediately preceded by a CONS
then it need not denote a list.
 
For example:
\begin{verbatim} 
[] is translated to the empty list NIL
[a] is translated to the 1-list (LIST a) or (CONS a NIL)
[:a] is translated to a
[a,b] is translated to the 2-list (LIST a b) or (CONS a (CONS b NIL))
[:a,b] is translated to (APPEND a (CONS b NIL))
[a,:b] is translated to (CONS a b)
[:a,:b] is translated to (APPEND a b)
[:a,b,c] is translated to (APPEND a (CONS b (CONS c NIL)))
[a,:b,c] is translated to (CONS a (APPEND b (CONS c NIL)))
[a,b,:c] is translated to (CONS a (CONS b c))
\end{verbatim} 

If the construct expression has iterators that control the production
of the list the resulting list depends on the form of the comma
expression.
i.e.
\begin{verbatim} 
construction ::= comma iteratortail
\end{verbatim} 

If the comma expression is recognised as denoting a list
by either preceding it by a colon, or having commas at top level
as above, then the successive values are appended.  If not then
the successive values are consed.
e.g.
\begin{verbatim} 
[f i for i in x] denotes the list formed by applying f to each
  member of the list x.
 
[:f i for i in 0..n] denotes the list formed by appending the
  lists f i for each i in 0..n.
\end{verbatim} 

\subsubsection{Patterns}
\label{sec:boot:pattern}

\begin{verbatim}  
is ::= arith | arith IS  pattern
\end{verbatim} 

The pattern in the proposition [[arith IS pattern]] has the same form
as the construct phrase without iterators. In this case, however it
denotes a class of lists rather than a list, and is composed
from identifiers rather than expressions.  The proposition
is translated into a program that tests whether the arith expression
denotes a list that belongs to the class. If it does then the value
of the is expression is true and the identifiers in
the pattern are assigned the values of the corresponding
components of the list. If the list does not match the pattern
the value of the is expression is false and the values of the
identifier might be changed in some unknown way that reflects the
partial success of the matching.
Because of this uncertainty,
it is advisable to use the variables in a pattern
as new definitions rather than assigning to variables that are
defined elsewhere.
\begin{verbatim}  
pattern::= identifier | constant | [ patternlist ]
\end{verbatim} 

The value of [[arith IS identifier]] is [[true]] and the value of 
[[arith]] is assigned to the [[identifier]].
[[(PROGN (SETQ identifier arith) T)]]
The expression [[arith IS constant]] is translated to 
[[(EQUAL constant arith)]].
The expression arith [[IS [ pattenlist ] ]]
produces a program which tests whether arith denotes a list
of the right length and that each patternitem matches the corresponding
list component.
 
\begin{verbatim} 
patternitem ::= EQ application  | DOT | pattern | name := pattern
\end{verbatim} 

If the [[patternitem]] is [[EQ application]] then the value is true if
the component is [[EQUAL]] to the value of the application expression.
If the [[patternitem]] is [[DOT]] then the value is [[true]] regardless of the
nature of the component.  It is used as a place-holder to test
whether the component exists.
If the patternitem is pattern then the component is matched against
the pattern as above.
If the [[patternitem]] is [[name:=pattern]] then the component is 
matched against 
the pattern as above, and if the value is [[true]] the component is assigned
to the name.  This last provision enables both a component and
its components to be given names.
\begin{verbatim}  
patternlist ::= listof(patternitem,COMMA)|
                listof(patternitem,COMMA) COMMA patterntail
                patterntail
 
patterncolon ::= COLON patternitem
 
patterntail ::= patterncolon |
               patterncolon COMMA listof(patternitem,COMMA)
\end{verbatim} 

The [[patternlist]] may contain one colon to indicate that the following
patternitem can match a list of any length. In this case
the matching rule is to construct the expression
with [[CONS]] and [[APPEND]] from the pattern as shown above and then test
whether the list can be constructed in this way, and if so
deduce the components and assign them to identifiers.
 
The effect of a pattern that occurs as a variable in a for iterator
is to filter the list by the pattern.
\begin{verbatim}  
forin ::= for pattern IN segment
\end{verbatim} 

is translated to two iterators
\begin{verbatim} 
          for g IN segment | g IS pattern
\end{verbatim} 
where [[g]] is an invented identifier.
\begin{verbatim} 
forin ::= for (name:=pattern) IN segment
\end{verbatim} 

is translated to two iterators
\begin{verbatim} 
          for name IN segment BAR name IS pattern
\end{verbatim} 

in order to both filter the list elements, and name both elements and
their components.
 
\subsubsection{Assignments}
\label{sec:boot:assignment}
 
A pattern may also occur on the left hand side of an assignment
statement, and has a slightly different meaning.
The purpose in this case is to give names to the components
of the list which is the value of the right hand side.
In this case no checking
is done that the list matches the pattern precisely and the only
effect is to construct the selectors that correspond to
the identifiers in the pattern, apply them to the value of the
right hand side and assign the selected components
to the corresponding identifiers.
The effect of applying [[CAR]] or [[CDR]] to arguments to which they are not
applicable will depend on the underlying Lisp system.
\begin{verbatim}  
assignment::= assignvariable BECOMES assignment| statement
 
assignvariable := OBRACK patternlist CBRACK | assignlhs
\end{verbatim} 

The assignment having a pattern as its left hand side is reduced
as explained above to one or more assignments having an identifier
on the left hand side.
The meaning of the assignment depends on whether the identifier
starts with a dollar sign or not, if it is and whether it is followed by
[[:local]] or [[:fluid]].
If the identifier does not start with a dollar sign it
is treated as local to the body of the function in which it
occurs, and
if it is not already an argument of the function,
a declaration to that effect is added to the Lisp code
by adding a [[PROG]] construction at top level within the body of the
function definition.  Note also the all local variables and fluid variables
are treated this way, resulting in initialization to [[nil]] before
execution of the body of the function.  Consequently care must be
exercised when assigning to Lisp special global variables.  If you
do not want that implicitly initialization to [[nil]], then use the
explicit [[SETQ]] Lisp special form in an application syntax.
 
If such an identifier assignment does not occur in the body
of a function but in a top level expression then
it is also treated as a local. The sole exception to this rule
is when the top level expression is an assignment to an identifier
in which case it is treated as global.
 
If the left hand side of an assignment is an identifier that starts with
a dollar sign it will not be classified as a local but will
be treated as non-local. If it is also followed by [[:local]] then it
will be treated as a declaration of a [[FLUID]] (VMLisp) or [[SPECIAL]]
variable (Common Lisp) which will be given an initial value which is the
value of the right hand side of the assignment statement.
The [[FLUID]] or [[SPECIAL]] variables may be referred to or assigned to
by functions that are applied in the body of the declaration.
 
If the left hand side of an assignment statement is
an identifier that does not start with a dollar sign followed
by [[:local]] then it will also be treated as a [[FLUID]] or [[SPECIAL]]
declaration, however it may only be assigned to in the body
of the function in which the assignment it occurs.
\begin{verbatim}  
assignment::= assignvariable BECOMES assignment | statement
 
assignvariable := OBRACK patternlist CBRACK | assignlhs
 
assignlhs::= name | name COLON local |
     name DOT primary DOT ... DOT primary
\end{verbatim} 

If the left hand side of an assignment has the form
\begin{verbatim}  
     name DOT primary DOT ... DOT primary
\end{verbatim} 
the assignment statement will denote an updating of some component
of the value of name.  In general [[name DOT primary := statement]]
will get translated to [[(SETELT name primary statement)]] or
[[(SETF (ELT name primary) statement)]]
There are however certain identifiers that denote components of
a list which will get translated to statements that update that
component (see appendix) e.g. 
\begin{verbatim} 
a.car:=b is translated to (SETF (CAR a) b) in Common Lisp.
\end{verbatim} 
The iterated [[DOT]] is used to update components of components
and e.g 

\begin{verbatim} 
a.b.c:=d is translated to (SETF (ELT (ELT a b)c) d)
 
exit::= assignment | assignment EXIT where
\end{verbatim} 

The exit format [[assignment EXIT where]] is used to give a value to
a blockof or pileof statements in which it occurs at top level.
 
The expression
\begin{verbatim} 
 (a =>b;c) will be translated to if a then b else c or
 (COND (a b) (T c))
\end{verbatim} 

If the exit is not a component of a blockof or pileof statements
then 
\begin{verbatim} 
a=>b will be translated to (COND (a b))
\end{verbatim}  

\subsubsection{Definitions}
 
Functions may be defined using the syntax
\begin{verbatim}  
functiondefinition::= name DEF where | name variable DEF where
 
 
variable ::= parenthesized variablelist | pattern
 
variableitem ::=
     name| pattern | name BECOMES pattern | name IS pattern
 
variablelist ::= variableitem | COLON name |
             variableitem COMMA variablelist
\end{verbatim} 

Function definitions may only occur at top level or after a [[where]].
The [[name]] is the name of the function being defined, and the
most frequently used form of the [[variable]] is either a single name
or a parenthesized list of names separated by commas.
In this case the translation to Lisp is straightforward, for example:
\begin{verbatim} 
f x == E  or f(x)==E is translated to (DEFUN f (x) TE)
f (x,y,z)==E is translated to (DEFUN f (x y z) TE)
f ()==E is translated to (DEFUN f () TE)
\end{verbatim} 

where [[TE]] is the translation of [[E]].
At top level 
\begin{verbatim} 
f==E is translated to (DEFUN f () TE)
\end{verbatim} 

The function being defined is that which when applied to its arguments
produces the value of the body as result where the variables
in the body take on the values of its arguments.
 
A pattern may also occur in the variable of a definition of a function
and serves the purpose, similar to the left hand side of assignments,
of naming the list components.
The phrase
\begin{verbatim} 
 name pattern DEF where
is translated to
        name g DEF (pattern:=g;where)
\end{verbatim} 

similarly
\begin{verbatim}  
 name1 name2 := pattern  DEF where  or name1 name2 is pattern  DEF where
 
are both translated to
        name1 name2 DEF (pattern:=name2;where)
\end{verbatim}  

similarly for patterns that occur as components of a list of
variables. order
\begin{verbatim} 
variablelist ::=
  variableitem | COLON name | variableitem COMMA variablelist
\end{verbatim} 

The parenthesized [[variablelist]] that occurs as a variable of a function
definition can contain variables separated by commas but can also
have a comma colon as its last separator.
 
This means that the function is applicable to lists of different
sizes and that only the first few elements corresponding to the
variables separated by commas are named, and
the last name after the colon denotes the rest of the list.
 
Macros may be defined only at top level, and must always have a variable
\begin{verbatim}  
macrodefinition::=  name variable MDEF where
\end{verbatim} 

The effect of a [[macrodefinition]] is to produce a Lisp macro
which is applied to arguments that are treated as expressions, rather
than their values, and whose result if formed by first substituting
the expressions for occurrences of the variables within the body
and then evaluating the resulting expression.
 
\subsubsection{Where Clauses}
\label{sec:boot:where-clause}
 
Expressions may be qualified by one or more function definitions
using the syntax
\begin{verbatim}  
where ::= exit | exit WHERE qualifier
 
qualifier ::= functiondefinition |
      pileof (functiondefinition) | blockof functiondefinition
\end{verbatim} 

The functions may only be used within the expression that is qualified.
This feature has to be used with some care, however, because
a where clause may only occur within a function body, and
the component functions are extruded, so to speak, from their contexts
renamed, and made into top level function definitions.
As a result the variables of the outer function cannot be referred to
within the inner function.
If a qualifying function has the format [[name DEF where]] then
the [[where]] phrase is substituted for all occurences of the name
within the expression qualified.
If an expression is qualified by a phrase that is not a
function definition then the result will be a compound statement
in which the qualifying phrase is followed by the qualified phrase.
 
\subsubsection{Tuples}
\label{sec:boot:tuples}

Although a tuple may appear syntactically
in any position occupied by a primary
it will only be given meaning when it is the argument to a function.
To denote a list it has to be enclosed in brackets rather than
parentheses. A tuple at top level is treated as if its components
appeared at top level in the order of the list.
\begin{verbatim}  
tuple::=   parenthesized (listof (where,COMMA))
\end{verbatim} 

\subsubsection{Blocks and Piles}
\label{sec:boot:block}

\begin{verbatim}  
block::=   parenthesized (listof (where,SEMICOLON))
pile::=    piled (listof (listof(where,SEMICOLON),BACKSET))
A block or a pile get translated to a compound statement or PROGN
\end{verbatim} 

\subsubsection{Top Level}
\label{sec:boot:top-level}

\begin{verbatim}  
toplevel ::=  functiondefinition | macrodefinition | primary
\end{verbatim} 

\subsubsection{Translation Functions}
\label{sec:boot:translation}

\begin{verbatim}  
(boottocl "filename")
translates the file "filename.boot" to 
the common lisp file "filename.clisp"
\end{verbatim} 

\begin{verbatim} 
(bootclam "filename")
translates the file "filename.boot" to 
the common lisp file "filename.clisp" 
\end{verbatim} 

producing, for each function a
hash table to store previously computed values indexed by argument
list.  The function first looks in the hash table for the result
if there returns it, if not computes the result and stores it in the
table.
 
\begin{verbatim}  
(boottoclc "filename")
translates the file "filename.boot" to
the common lisp file "filename.clisp" 
with the original boot code as comments
\end{verbatim} 
 
\begin{verbatim} 
(boot "filename")
translates the file "filename.boot" to
the common lisp file "filename.clisp", 
compiles it to the file  "filename.bbin" 
and loads  the bbin file.
\end{verbatim} 

\begin{verbatim} 
(bo "filename")
translates the file "filename.boot"
and prints the result at the console
\end{verbatim} 

\begin{verbatim} 
(stout "string") translates the string "string"  
and prints the result at the console
\end{verbatim} 
 
\begin{verbatim} 
(sttomc "string") translates the string "string"  
to common lisp, and compiles the result.
\end{verbatim} 
 
\begin{verbatim} 
(fc "functionname" "filename")
attempts to find the boot function
functionname in the file filename, 
if found it translates it to common
lisp, compiles and loads it.
\end{verbatim} 
 
\begin{verbatim} 
BOOT_-COMPILE_-DEFINITION_-FROM_-FILE(fn,symbol)
 is similar to fc, fn is the file name but symbol is the  symbol
 of the function name rather than the string.
(fn,symbol)
\end{verbatim} 
 
\begin{verbatim} 
BOOT_-EVAL_-DEFINITION_-FROM_-FILE(fn,symbol)
attempts to find the definition of symbol in file fn, but this time
translation is followed by EVAL rather than COMPILE
\end{verbatim} 
 
\begin{verbatim} 
(defuse "filename")
Translates the file filename, and writes a report of the
functions defined and not used, and used and not defined in the
file filename.defuse
\end{verbatim} 
 
\begin{verbatim} 
(xref "filename")
Translates the file filename, and writes a report of the
names  used, and  where used to the file filename.xref
\end{verbatim}

\subsection{Reserved identifiers}
\label{sec:boot:reserved-identifiers}

The following identifiers are reserved by Boot.
\begin{verbatim}
  and    append   apply     atom      car    cdr       cons      copy
  croak  drop     exit      false     first  function  genvar    IN
  is     isnt     lastNode  LAST      list   member    mkpf      nconc
  nil    not      NOT       nreverse  null   or        otherwise PAIRP
  removeDuplicates          rest      reverse          setDifference
  setIntersection setPart   setUnion  size   strconc   substitute
  take   true     PLUS      MINUS     TIMES  POWER     SLASH     LT
  GT     LE       GE        SHOEEQ    NE     T
\end{verbatim}

The following identifiers designate special accessor functions in Boot.
\begin{verbatim}
  setName     setLabel    setLevel   setType    setVar    setLeaf
  setLeaf     setDef      aGeneral   aMode      aTree     aValue
  attributes  cacheCount  cacheName  cacheReset cacheType env
  expr        CAR         mmCondition           mmDC      mmImplementation
  mmSignature mmTarget    mode       op         opcode    opSig
  CDR         sig         source     streamCode streamDef streamName
  target
\end{verbatim}


\section{The Makefile}
\label{sec:Makefile}

When all of the native object files are produced we construct a
lisp image that contains the boot translator, called [[bootsys]], which
lives in the [[$(axiom_target_bindir)]] directory.  This [[bootsys]] image
is critical for the rest of the makefiles to succeed.
 
There are two halves of this file. the first half compiles the .lisp files
that live in the src/boot directory. the second half compiles the .clisp
files (which are generated from the .boot files). It is important that
the .clisp files are kept in the src/boot directory for the boot translator
as they cannot be recreated without a boot translator (a bootstrap problem).
 
An important subtlety is that files in the boot translator depend on the
file npextras. there are 3 macros in npextras that must be in the lisp
workspace (\verb$|shoeOpenInputFile| |shoeOpenOutputFile| memq$). 

\subsection{Environment}
\label{sec:Makefile:env}

\subsubsection{Lisp Images}
\label{sec:Makefile:env:lisp-images}

We will use create and use several lisp images during the build
process. We name them here for convenience. 

\paragraph{[[AXIOM_LOCAL_LISP]].}  First we create a Lisp image
that contains at least three macros  for translating
Boot source files.  We do this by loading \File{initial-env.lisp}
in [[AXIOM_LISP]], and saving the resulting image.  That image is then
used to build the bootstrapping Boot translator.
<<environment>>=
AXIOM_LOCAL_LISP_sources = initial-env.lisp
AXIOM_LOCAL_LISP = $(builddir)/local-lisp$(EXEEXT)
@

\paragraph{[[BOOTSYS_FOR_TARGET]].}
The [[$(BOOTSYS_FOR_TARGET)]] image is the final Boot translator image,
produced after several bootstrap stages.  That is the result of
running the \Tool{Make} target [[all-boot]].
<<environment>>=
BOOTSYS_FOR_TARGET = $(axiom_target_bindir)/bootsys$(EXEEXT)
@ 


\section{Proclaim optimization}
\label{sec:proclaim}

GCL, and possibly other common lisps, can generate much better
code if the function argument types and return values are proclaimed.

In theory what we should do is scan all of the functions in the system
and create a file of proclaim definitions. These proclaim definitions
should be loaded into the image before we do any compiles so they can
allow the compiler to optimize function calling.

GCL has an approximation to this scanning which we use here. 

The first step is to build a version of GCL that includes gcl\_collectfn.
This file contains code that enhances the lisp compiler and creates a
hash table of structs. Each struct in the hash table describes information
that about the types of the function being compiled and the types of its
arguments. At the end of the compile-file this hash table is written out
to a ".fn" file. 

The second step is to build axiom images (depsys, interpsys, AXIOMsys)
which contain the gcl\_collectfn code.

The third step is to build the system. This generates a .fn file for 
each lisp file that gets compiled.

The fourth step is to build the proclaims.lisp files. There is one
proclaims.lisp file for 
boot (boot-proclaims.lisp), 
interp (interp-proclaims.lisp), and 
algebra (algebra-proclaims.lisp).

To build the proclaims file (e.g. for interp) we:
\begin{verbatim}
(a) cd to obj/linux/interp
(b) (yourpath)/axiom/obj/linux/bin/lisp
(c) (load "sys-pkg.lsp") 
(d) (mapcar #'load (directory "*.fn"))
(e) (with-open-file (out "interp-proclaims.lisp" :direction :output) 
      (compiler::make-proclaims out))
\end{verbatim}
Note that step (c) is only used for interp, not for boot.

The fifth step is to copy the newly constructed proclaims file back
into the src/interp diretory (or boot, algebra).

In order for this information to be used during compiles we define
<<environment>>=
PROCLAIMS=(load "$(srcdir)/boot-proclaims.lisp")

@

\section{Special Commands}
\label{sec:special-commands}

We are working in a build environment that combines Makefile
technology with Lisp technology. Instead of invoking a command
like {\bf gcc} and giving it arguments we will be creating 
Lisp S-expressions and piping them into a Lisp image. The
Lisp image starts, reads the S-expression from standard input,
evaluates it, and finding an end-of-stream on standard input, exits.


\section{The Boot Compiler}
\label{sec:boot-compiler}

This section describes the set of object files that make the Boot compiler.

\subsection{The Bootstrap files}
\label{sec:boot-compiler:bootstrap}

This is a list of all of the files that must be loaded to construct the
boot translator image. 
<<environment>>= 
boot_objects = $(boot_sources:.boot=.$(FASLEXT))

boot_SOURCES = $(addsuffix .pamphlet, $(boot_sources))

pamphlets = Makefile.pamphlet $(AXIOM_LOCAL_LISP_SOURCES) $(boot_SOURCES)
@

[[$(boot_sources)]] is a list of the boot file targets. If you modify a
boot file you'll have to explicitly build the clisp files and
merge the generated code back into the pamphlet by hand. The
assumption is that if you know enough to change the fundamental
bootstrap files you know how to migrate the changes back.
This process, by design, does not occur automatically (though it
could).

The Boot compiler, [[bootsys]], is built from a set of source files
written in Boot.  Note that the order is 
important as earlier files will contain code needed by later files.
<<environment>>=
boot_sources = ptyout.boot btincl2.boot \
	btscan2.boot typrops.boot btpile2.boot \
	typars.boot tytree1.boot

boot_clisp = $(boot_sources:.boot=.clisp)
boot_data = $(boot_sources:.boot=.data)
boot_fn = $(boot_sources:.boot=.fn)
@
These source files use macros defined in the first set, and they be compiled
in an environment where those macros are present.



The Boot source file for [[bootsys]] are automatically extracted --- 
only during bootstrap --- from the pamphlets into the current build
directory.  When bootstrapping, they are the inputs to the stage0, stage1
 [[bootsys]] compilers.

<<boot from pamphlet>>=
.PRECIOUS: %.boot
%.boot: $(srcdir)/%.boot.pamphlet
	$(axiom_build_document) --tangle $< 
@

Since the Boot language is defined as a syntactic sugar over Lisp 
(a reasonably tasty sugar), the
the second set of source files (written in Boot) is first translated 
to Lisp, and the result of that translation is subsequently compiled to
native object files.

Partly for bootstrapping reasons, and partly because Axiom (therefore
Boot) is not yet widespread, the pamphlets for the source files written 
in Boot currently keep a cache of their translated versions.  Hopefully
the maintainance of that cache will be unnecessary as the build machinery
becomes more and more improved, and Axiom gets in widespread use.
<<environment>>=
boot_cached_clisp = $(boot_sources:.boot=.clisp)
@

\section{Bootstrapping Boot}
\label{sec:bootstrapping}

When the system is configured for bootstrap, we build the Boot compiler ---
[[bootsys]] --- in three steps:
\begin{enumerate}
\item a stage-0 Boot compiler, built from the cached (Lisp) source files;

\item a stage-1 Boot compiler, built the original Boot source files using the
  stage-0 Boot compiler; 

\item and a stage-2 Boot compiler, built from original Boot source files 
  using the stage-2 Boot compiler.
\end{enumerate}
Notice that in last two steps, the source file written in Boot are first
translated to Lisp using the freshly built Boot compiler, and the resulting
Lisp files subsequently compiled to natve object files.

Ideally, we should also compare the intermediate Lisp source files from 
stage 1 and 2 to detect possible miscompilation.  We don't do that 
for the moment.

\subsection{Compiling the Boot source files}
\label{sec:bootstrapping:source-files}

We compile the Boot compiler source files written in Boot only 
at stage 1 and 2 (when bootstrapping).  As explained earlier, the
compilation of these files proceeds in two steps:
\begin{enumerate}
\item Translate the Boot source files to Lisp code, 
\item compile the resulting Lisp source files to native object code.
\end{enumerate}
These files depend on macros defined in the Lisp source files without
dependency (see previous section).
<<compile Boot files from pamphlets>>=
.PRECIOUS: %.$(FASLEXT)
.PRECIOUS: %.clisp
.PRECIOUS: %.boot

$(boot_objects): %.$(FASLEXT): %.clisp
	$(COMPILE_LISP) --use=$(AXIOM_LOCAL_LISP) $<

%.clisp: %.boot
	$(BOOT_TO_LISP)

<<boot from pamphlet>>
@

\subsection{Building [[bootsys]]}
\label{sec:bootstrapping:build-bootsys}

The Boot compiler [[bootsys]] image is constructed as the
result of loading the needed object files in [[AXIOM_LOCAL_LISP]].
and saving the resulting image back on disk.
<<build bootsys>>=
$(AXIOM_LOCAL_LISP): $(AXIOM_LOCAL_LISP_sources)
	$(axiom_build_document) --tag=lisp --mode=save-image --output=$@ \
		--use=$(LISPSYS) $(AXIOM_LOCAL_LISP_sources)

bootsys$(EXEEXT): $(OBJECTS)
	$(axiom_build_document) --tag=lisp --mode=link --output=$@ \
		--use=$(AXIOM_LOCAL_LISP) '$(patsubst %, "%", $(OBJECTS))'
@

The make rule [[bootsys]] is executed through a recursive call to [[$(MAKE)]]
from the various bootstrapping stage.  The recursive call supplies
values for [[OBJECTS]].

\subsection{The various bootstrapping stages}
\label{sec:bootstrapping:stages}

\subsubsection{Stage 0}
\label{sec:bootstrapping:stages:stage-0}

We build the stage-0 Boot compiler from the cached Lisp souces code.
<<stage 0 boot compiler>>=
.PRECIOUS: stage0/%.clisp
.PRECIOUS: stage0/%.$(FASLEXT)

stage0_boot_clisp = $(addprefix stage0/, $(boot_clisp))

stage0_boot_objects = $(addprefix stage0/, $(boot_objects))

stage0/bootsys$(EXEEXT): 
	@echo Building stage 0
	[ -d stage0 ] || $(mkinstalldirs) stage0
	rm -rf prev-stage
	rm -f $(stage0_boot_objects)
	rm -f $(stage0_boot_clisp)
	$(MAKE) $(AX_FLAGS) OBJECTS="$(stage0_boot_objects)" bootsys$(EXEEXT)
	mv bootsys$(EXEEXT) stage0
	$(LN_S) stage0 prev-stage

$(stage0_boot_objects): $(AXIOM_LOCAL_LISP)

stage0/%.$(FASLEXT): stage0/%.clisp
	$(COMPILE_LISP)  --use=$(AXIOM_LOCAL_LISP) $<

stage0/%.clisp: $(srcdir)/%.boot.pamphlet
	$(axiom_build_document) --tangle=$*.clisp --output=$@ $<
@

\subsubsection{Stage 1}
\label{sec:bootstrapping:stages:stage-1}

<<stage 1 boot compiler>>=
stage1/bootsys$(EXEEXT): stage0/bootsys$(EXEEXT)
	@echo Building stage 1
	[ -d stage1 ] || $(mkinstalldirs) stage1
	$(MAKE) $(AX_FLAGS) OBJECTS="$(boot_objects)" bootsys$(EXEEXT)
	-rm -f $(addprefix stage1/, $(boot_objects))
	-rm -f $(addprefix stage1/, $(boot_clisp))
	mv $(boot_objects) stage1
	mv $(boot_clisp) stage1
	-mv $(boot_data) $(boot_fn) stage1
	mv bootsys$(EXEEXT) stage1
	-rm -rf prev-stage
	$(LN_S) stage1 prev-stage
@

\subsubsection{Stage 2}
\label{sec:bootstrapping:stages:stage-2}

<<stage 2 boot compiler>>=
stage2/bootsys$(EXEEXT): stage1/bootsys$(EXEEXT)
	@echo Building stage 2
	[ -d stage2 ] || $(mkinstalldirs) stage2
	$(MAKE) $(AX_FLAGS) OBJECTS="$(boot_objects)" bootsys$(EXEEXT)
	-rm -rf prev-stage
	-rm -f $(addprefix stage2/, $(boot_objects))
	-rm -f $(addprefix stage2/, $(boot_clisp))
	mv $(boot_objects) stage2
	mv $(boot_clisp) stage2
	-mv $(boot_data) $(boot_fn) stage2
	mv bootsys$(EXEEXT) stage2
@

<<bootstrap>>=
<<stage 0 boot compiler>>

<<stage 1 boot compiler>>

<<stage 2 boot compiler>>
@


\section{Making the documentation}
\label{sec:doc}

\subsection{Compiling Lisp files without deps from pamphlets}
<<initial-env.lisp>>=
.PRECIOUS: %.lisp

initial-env.lisp: $(srcdir)/initial-env.lisp.pamphlet
	$(axiom_build_document) --tangle $<
@

\subsection{boot from pamphlet}
<<boot from pamphlet>>=
.PRECIOUS: %.boot

%.boot: $(srcdir)/%.boot.pamphlet
	$(axiom_build_document) --tangle $<
@


\section{Making the documentation}
<<environment>>=

LISPSYS= $(axiom_build_bindir)/lisp

COMPILE_LISP = \
	$(axiom_build_document) --tag=lisp --mode=compile --output=$@

BOOT_TO_LISP = \
	$(axiom_build_document) --tag=boot --mode=translate \
	--use=./prev-stage/bootsys $<
@

\section{Cleanup}
<<cleanup>>=
mostlyclean-local:
	@rm -f $(AXIOM_LOCAL_LISP)
	@rm -f $(BOOTSYS_FOR_TARGET)
	@rm -rf prev-stage
	@rm -rf stage0 stage1 stage2
	@rm -f *.data *.fn
	@rm -f stamp

clean-local: mostlyclean-local
	@rm -f $(boot_sources)
	@rm -f *.lisp

distclean-local: clean-local
@


\section{Global variables}

The Boot implementation uses a number of global variables 
for communication between several routines.  Some of them follow
the syntactic convention of starting their names with [[$]].  Some
don't.

\subsection{[[$linepos]]}

\subsection{[[$f]]}

\subsection{[[$r]]}

\subsection{[[$ln]]}

\subsection{[[$sz]]}

\subsection{[[$n]]}

\subsection{[[$floatok]]}

\subsection{[[$bfClamming]]}

\subsection{[[$GenVarCounter]]}

\subsection{[[$inputstream]]}

\subsection{[[$stack]]}

\subsection{[[$stok]]}

\subsection{[[$ttok]]}

\subsection{[[$op]]}

\subsection{[[$wheredefs]]}

\subsection{[[$typings]]}

\subsection{[[$returns]]}

\subsection{[[$bpCount]]}

\subsection{[[$bpParentCount]]}

\subsection{[[$lispWordTable]]}

\subsection{[[$bootUsed]]}

\subsection{[[$bootDefinedTwice]]}

\subsection{[[$used]]}

\subsection{[[$letGenVarCounter]]}

\subsection{[[$isGenVarCounter]]}

\subsection{[[$inDefIS]]}

\subsection{[[$fluidVars]]}

\subsection{[[$locVars]]}

\subsection{[[$dollarVars]]}




\section{The Makefile}
<<*>>=
<<environment>>

subdir = src/boot/
 
# this stanza will create the final bootsys image
all: all-ax

all-ax: stage2/bootsys$(EXEEXT)
	$(INSTALL_PROGRAM) stage2/bootsys$(EXEEXT) $(axiom_build_bindir)

<<bootstrap>>

<<build bootsys>>

<<compile Boot files from pamphlets>>
<<initial-env.lisp>>

<<cleanup>>
@

\eject
\begin{thebibliography}{99}
\bibitem{1} \$SPAD/src/boot/boothdr.lisp.pamphlet
\bibitem{2} \$SPAD/src/boot/btincl2.boot.pamphlet
\bibitem{3} \$SPAD/src/boot/btpile2.boot.pamphlet
\bibitem{4} \$SPAD/src/boot/btscan2.boot.pamphlet
\bibitem{5} \$SPAD/src/boot/exports.lisp.pamphlet
\bibitem{6} \$SPAD/src/boot/npextras.lisp.pamphlet
\bibitem{7} \$SPAD/src/boot/ptyout.boot.pamphlet
\bibitem{8} \$SPAD/src/boot/typars.boot.pamphlet
\bibitem{9} \$SPAD/src/boot/typrops.boot.pamphlet
\bibitem{10} \$SPAD/src/boot/tytree1.boot.pamphlet
\end{thebibliography}
\end{document}
